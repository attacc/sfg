%\documentclass[twocolumn,showpacs,prl,superscriptaddress,aps,floatfix]{revtex4-1}
%\documentclass[preprint,showpacs,prb,superscriptaddress,aps,floatfix]{revtex4-1}
\documentclass[submission, Phys]{SciPost}


\usepackage{bm}
\usepackage{xcolor}
\usepackage{graphicx}
\usepackage{lineno}
\usepackage{comment}
\usepackage[caption=false]{subfig} % <=== "caption=false" added
\usepackage{hyperref}
\usepackage{url}
\usepackage{amsmath,mleftright,mathtools}
\usepackage{mathrsfs}

\DeclareMathAlphabet\mathbfcal{OMS}{cmsy}{b}{n}

\newcommand{\torvergata}{University of Rome Tor Vergata, Rome, Italy}
\newcommand{\piim}{Universit\'e Aix-Marseille, Laboratoire de Physique des Interactions Ioniques et Moléculaires (PIIM), UMR CNRS 7345, F-13397 Marseille, France}
\newcommand{\cinam}{CNRS/Aix-Marseille Universit\'e, Centre Interdisciplinaire de Nanoscience de Marseille UMR 7325 Campus de Luminy, 13288 Marseille cedex 9, France}
\newcommand{\etsf}{European Theoretical Spectroscopy Facilities (ETSF)}

\newcommand{\lv}{{\bf a}}
\newcommand{\bb}{{\bf b}}
\newcommand{\uu}{{\bf u}}
\newcommand{\qq}{{\bf q}}
\newcommand{\hh}{{\bf h}}
\newcommand{\rr}{{\bf r}}
\newcommand{\pp}{{\bf p}}
\newcommand{\PP}{{\bf P}}
\newcommand{\kk}{{\bf k}}
\newcommand{\HH}{{\bf H}}
\newcommand{\GG}{{\bf G}}
\newcommand{\SiS}{{\bf \Sigma}}
\newcommand{\VV}{{\bf V}}
\newcommand{\UU}{{\bf U}}
\newcommand{\w}{\omega}
\newcommand{\tf}{\textbf}
\newcommand{\bo}{\mathbf}
\newcommand{\br}{{\bf r}}
\newcommand{\be}{\begin{equation}}
\newcommand{\ee}{\end{equation}}
\newcommand{\ben}{\begin{equation*}}
\newcommand{\een}{\end{equation*}}
\newcommand{\bea}{\begin{eqnarray}}
\newcommand{\eea}{\end{eqnarray}}
\newcommand{\bean}{\begin{eqnarray*}}
\newcommand{\eean}{\end{eqnarray*}}
\newcommand{\nup}{n_{\uparrow}}
\newcommand{\ndown}{n_{\downarrow}}
\newcommand{\Id}[1] {\int \! \! {\rm d}^3 #1}
\renewcommand{\v}[1]{{\bf #1}}
\renewcommand{\[}{\left[}
\renewcommand{\]}{\right]}
\renewcommand{\(}{\left(}
\renewcommand{\)}{\right)}
\def\efield{\mathbf{\cal E}} 
\def\ket#1{\vert#1\rangle}
\def\bra#1{\langle#1\vert}
\def\susc#1{\chi^{(#1)}}
\def\ket#1{\vert#1\rangle}
\def\bra#1{\langle#1\vert}
\def\ai{\emph{ab-initio}\ }


\begin{document}

%\title{Excitons under strain: light absorption and emission \\ in strained hexagonal boron nitride}

\begin{center}{\Large \textbf{Sum-frequency generation}}\end{center}

\begin{center}

Mike Pionteck, Simone Sanna, Claudio Attaccalite\textsuperscript{3,4}
\end{center}

%\author{P. Lechifflart}
%\affiliation{\cinam}
%\author{F. Paleari}
%\affiliation{\cnr}
%\affiliation{\cnrmodena}
%\author{C. Attaccalite}
%\affiliation{\cinam}
%\affiliation{\etsf}


\begin{center}
{\bf 1} \torvergata
\\
{\bf 2} \piim
\\
{\bf 3} \cinam
\\
{\bf 4} \etsf
\\
% TODO: provide email address of corresponding author
\end{center}

\date{\today}

\begin{abstract}

\end{abstract}

\vspace{10pt}
\noindent\rule{\textwidth}{1pt}
\tableofcontents\thispagestyle{fancy}
\noindent\rule{\textwidth}{1pt}
\vspace{10pt}


\section{Analysis of the respone}
In these notes we generalized the approch developed in Ref.~\cite{Attaccalite2013} to more than one external field. Here we consider the perturbation of the laser fields with frequencies $\omega_1$ and $\omega_2$.\\
We run the simulations for a time much larger than the dephasing time $1/\gamma_{\text{deph}}$ and sample $\PP(t)$ close to the end of the simulation, as explained in Ref.~\cite{Attaccalite2013}.
Since $\gamma_{\text{deph}}$ determines also the spectral broadening, we cannot choose it arbitrary small. For example in the present calculations we have chosen $1/\gamma_{\text{deph}}$ of 6 fs that corresponds to a broadening of approximately 0.2 eV (comparable with the experimental one) and thus we run the simulations for 50-55 fs.\\
Once all the eigenfrequencies of the system are filtered out, the remaining polarization $\PP(t)$ is a periodic function of period $T_L$ equal to the least common multiple of $T_1=\frac{2\pi}{\omega_1}$ and  $T_2=\frac{2\pi}{\omega_2}$  , where $\omega_1,\omega_2$ are the frequencies of the external perturbations and can be expanded in a Fourier series as
\be\label{eq:frrexp}
\PP(t) = \sum_{n,m=-\infty}^{+\infty} \pp_{n,m} e^{-i(\omega_n+\omega_m) t},
\ee  
with $\omega_n = n \omega_1$ and $\omega_m = m \omega_2$, and complex coefficients:
\begin{equation}\label{eq:frrcff}
	\pp_{n,m}  = \mathscr{F}\{\PP(\omega_n+\omega_m)\} =\int_{0}^{T_L} dt \PP(t) e^{i(\omega_n + \omega_m) t}.
\end{equation}
To obtain the optical susceptibilities of order $n+m$ at frequency $\omega_1+\omega_2$ one needs to calculate the $\pp_{n,m}$ of Eq.~\eqref{eq:frrexp}, proportional to $\susc{n+m}$ by the $n$-th power of the $\efield_1$ and $m$-th power of the $\efield_2$. 
However, the expression in Eq.~\eqref{eq:frrcff} is not the most computationally convenient since one needs a very short time step---significantly shorter than the one needed to integrate the EOMs---to perform the integration with sufficient accuracy. As an alternative we use directly Eq.~\eqref{eq:frrexp}: we truncate the Fourier series to an order $S$ larger than the one of the response function we are interested in. We sample $(2S+1)^2$ values $\PP_i\equiv\PP(t_i)$ within a period $T_L$ if possible or otherwise the largest period between $T_1$ and $T_2$, as illustrated in Figure~\ref{fg:ptanalysis}. Then Eq.~\eqref{eq:frrexp} reads as a system of linear equations 
\be
{\cal F}_{i(n,m)} p^\alpha_{n,m} = P^\alpha_i,
\label{eq:fouinv}
\ee 
from which the component $p^\alpha_{n,m}$ of $\pp_{n,m}$ in the $\alpha$ direction is found by inversion of the $(2S+1)^2\times(2S+1)^2$ Fourier matrix ${\cal F}_{i(n,m)} \equiv \exp[-i(\omega_n +\omega_m)t_i]$. We found that in presence of more than one field the matrix could be degenerate when field have comensurable frequencies and so on, in these cases we used single-value decomposition to construct the pseudo-inverse that is used to estimated the non-linear coefficents. We found in general that the second harmonic generation converges with S equal to 4 while the third harmonic requires S equal to 6.
\subsection{Comparison with the previous code}
Suppose now we perturb the system with two laser with the same frequency $\omega_1 = \omega_2 =\omega_L$ with the same intensity. From eq.~\ref{eq:frrexp} we want to reconstruct the coefficent of the second harmonic generation. In this case we isolate all terms that oscillate with $2\omega_L$
\be
\PP^{2\omega_L}(t) =\left [ \pp_{2,0} + \pp_{0,2} +\pp_{1,1} \right ] e^{-i(2\omega_L) t} + O(E^4),
\ee
therefore we should expect that standard coefficent of the SHG will be given by $\pp_{2}= \pp_{2,0} + \pp_{0,2} +\pp_{1,1}$.
\section{Conclusions}
%\section*{Acknowledgments}
\section*{Acknowledgments}
C.A. acknowledge B. Demoulin and  A. Saul for the management of the computer cluster \emph{Rosa}. ADD Acknowledgments project COLIBRI. 
%\addcontentsline{toc}{chapter}{Bibliography}
%\bibliographystyle{apsrev4-1}
\bibliography{shg_2d.bib,tpa.bib,gase.bib}
\nolinenumbers
\end{document}
